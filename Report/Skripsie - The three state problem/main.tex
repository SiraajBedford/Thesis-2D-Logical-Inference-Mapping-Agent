\documentclass{article}
\usepackage[utf8]{inputenc}
\usepackage{graphicx}
\usepackage{float}
\usepackage{listings}
\usepackage{xcolor}

\definecolor{codegreen}{rgb}{0,0.6,0}
\definecolor{codegray}{rgb}{0.5,0.5,0.5}
\definecolor{codepurple}{rgb}{0.58,0,0.82}
\definecolor{backcolour}{rgb}{0.95,0.95,0.92}

\lstdefinestyle{mystyle}{
    backgroundcolor=\color{backcolour},   
    commentstyle=\color{codegreen},
    keywordstyle=\color{magenta},
    numberstyle=\tiny\color{codegray},
    stringstyle=\color{codepurple},
    basicstyle=\ttfamily\footnotesize,
    breakatwhitespace=false,         
    breaklines=true,                 
    captionpos=b,                    
    keepspaces=true,                 
    numbers=left,                    
    numbersep=5pt,                  
    showspaces=false,                
    showstringspaces=false,
    showtabs=false,                  
    tabsize=2
}

\lstset{style=mystyle}

\usepackage[
top    = 1cm,
bottom = 2cm,
left   = 1cm,
right  = 1cm]{geometry}

\title{Skripsie sub-problem:  The three state environment}
\author{21093741 }
\date{August 2020}

\begin{document}

\maketitle

\centerline{\rule{8in}{0.01in}}

\section*{Introduction}

This document aims to solve the problem for a simple 3 block environment for a logical agent to learn using inference by model checking.

\section{Definition of Atomic Sentences}

$\mathcal{P}_t$-To left is open at time $t$\\
$\neg \mathcal{P}_t$-To left is not open at time $t$\\
$\mathcal{Q}_t$-To right is open at time $t$\\
$\neg \mathcal{Q}_t$-To right is not open at time $t$\\
$\mathcal{R}_t$-Agent chose to move right at time $t$\\
$\mathcal{S}_t$-Agent chose to move left at time $t$\\
$\mathcal{T}_t$-In state 1 (In the left most block) at time $t$\\
$\mathcal{U}_t$-In state 2 (The middle block) at time $t$\\
$\mathcal{V}_t$-In state 3 (In the right most block) at time $t$\\




\section{Logical Connectives}
 


$\neg$ - not\\
$\wedge$ - and\\
$\vee$ - or\\
$\Rightarrow$ - implies\\
$\Leftrightarrow$ - if and only if\\

Operator precedence - $\neg$, $\wedge$, $\vee$, $\Rightarrow$, $\Leftrightarrow$

These can be used to connect atomic sentences to form complex sentences. The complex sentences can be used as prior knowledge in the initial knowledge base.


\section{State definition in the Knowledge Base}


\subsection{State 1}

\begin{figure}[H]
    \centering
    \includegraphics[scale=0.4]{Figures/Simple Agent.png}
    \caption{Agent in State 1}
    \label{fig:S1}
\end{figure}

\begin{itemize}
    \item Option 1: $ \mathcal{T}_{t-1} \wedge S_t \Rightarrow \mathcal{T}_t $, If in state 1 initially and agent moves left, agent stays in state 1.
    \item Option 2: $ \mathcal{U}_{t-1} \wedge S_t \Rightarrow \mathcal{T}_t $, If in state 2 initially and agent moves left, agent is in state 1.
    \item $\neg \mathcal{P}_t \wedge \mathcal{Q}_t \Rightarrow \mathcal{T}_t$. At the current time step, if the agent observes a closed space to the left and open space to the right as a result of performing options 1 or 2 then agent is in state 1.   
\end{itemize}




\subsection{State 2}


\begin{figure}[H]
    \centering
    \includegraphics[scale=0.4]{Figures/Simple Agent (1).png}
    \caption{Agent in State 2}
    \label{fig:S2}
\end{figure}

\begin{itemize}
    \item Option 1: $ \mathcal{T}_{t-1} \wedge R_t \Rightarrow \mathcal{U}_t $, If in state 1 initially and agent moves right, agent is in state 2.
    \item Option 2: $ \mathcal{V}_{t-1} \wedge S_t \Rightarrow \mathcal{U}_t $, If in state 3 initially and agent moves left, agent is in state 2.
    \item $ \mathcal{P}_t \wedge \mathcal{U}_t \Rightarrow \mathcal{T}_t$. At the current time step, if the agent observes an open space to the left and open space to the right as a result of performing options 1 or 2 then agent is in state 2.   
\end{itemize}



\subsection{State 3}


\begin{figure}[H]
    \centering
    \includegraphics[scale=0.4]{Figures/Simple Agent (2).png}
    \caption{Agent in State 3}
    \label{fig:S3}
\end{figure}

\begin{itemize}
    \item Option 1: $ \mathcal{V}_{t-1} \wedge R_t \Rightarrow \mathcal{V}_t $, If in state 3 initially and agent moves right, agent stays in state 3.
    \item Option 2: $ \mathcal{U}_{t-1} \wedge R_t \Rightarrow \mathcal{V}_t $, If in state 2 initially and agent moves right, agent is in state 3.
    \item $\neg \mathcal{Q}_t \wedge \mathcal{P}_t \Rightarrow \mathcal{V}_t$. At the current time step, if the agent observes an open space to the left and closed space to the right as a result of performing options 1 or 2 then agent is in state 3.   
\end{itemize}

\section{Inference by Model Checking of possible worlds}

The point of inference is to find out which state the agent is in. We can do this by referencing rules in the KB. 










\section{Design of agent, environment, inference engine and knowledge base}

\subsection{Agent requirements}

\begin{itemize}
    \item AR1: Should be able to observe the blocks to the left and right of the agent at time $t$.
    \item AR2: Should be able to know if the blocks to the left or right is a wall (1) or an open space (0).
    \item AR3: Should be able to move right.
    \item AR4: Should be able to move left.
    \item AR5: Should be able to move randomly for inference.
\end{itemize}


\subsection{Environment requirements}

\begin{itemize}
    \item ER1: Should have open blocks for 3 states. (left, middle, right)
    \item ER2: Should have boundary blocks to prevent indexing errors.
    \item ER3: Read in the map as CSV or generate map using a function.
\end{itemize}


\subsection{Knowledge base requirements}

\begin{itemize}
    \item KR1: Should be able to store logic rules. (Will probably use strings to accomplish this).
    \item KR2: Should be able to add new logic rules if not in the KB.
    \item KR3: Should interpret logic the same way the logic and inference engine performs logical operations.
\end{itemize}

\subsection{Inference engine requirements}

\begin{itemize}
    \item IR1: Should be able to generate all possible worlds
    \item IR2: Check whether the facts in the KB agree with possible worlds by querying the KB.
    \item IR3: Should be able to execute the following logical operations:  $\neg$, $\wedge$ , $\vee$, $\Rightarrow$, $\Leftrightarrow$
    \item IR4: Use recursion operation from one time step to the next.
\end{itemize}



%%%%%%%%%%%%%%%%%%%%%%%%%%%%%%%%%%%%%%%%%%%%%%%%%%%%%%%%%%%%%%%%%%%%%%%%%%%%%%%%%%%%%%%%%%%%%%%%%%%

\section{Implementing code in python using the design requirements}

\subsection{Agent}

\begin{itemize}
    \item AR1 and AR2: method that takes in current position of agent in environment and returns left and right squares and if these squares are walls (1) or open spaces (0).
    \item AR3 and AR4: The agent has \textit{self} attribute that has an $x$ coordinate. This can be increased or decreased by 1 to let the agent move itself.
    \item AR5: Use random module in python for choosing right or left.
\end{itemize}


\begin{lstlisting}[language=Python, caption=Agent class outline]
#Agent

import random # for choosing left or right action
import numpy # for interpreting maps as numpy arrays
import environment #design this library for getting
    
class Simple_Agent():
    
    def __init__(self, x, obs):
        # initial x position will be given for prototyping phase
        # get left and right observations from environment module.
        
    def move_left(self):
        # decrease horizontal agent position by 1.
    
    def move_right(self):
        # increment horizontal agent position by 1.
        
    def random_move(self):
        # choose to move right or left randomly.
        
    #def observe(self):
        #get left and right blocks from environment module. Give this to the agent
\end{lstlisting}


%%%%%%%%%%%%%%%%%%%%%%%%%%%%%%%%%%%%%%%%%%%%%%%%%%%%%%%%%%%%%%%%%%%%%%%%%%%%%%%%%%%%%%%%%%%%%%%%%%%


\subsection{Environment}

\begin{itemize}
    \item ER1: Environment matrix has three zero elements.
    \item ER2: Environment matrix has a 'one' element on borders.
    \item ER3: Have functions to read in .csv or return numpy array from function
\end{itemize}

\begin{lstlisting}[language=Python, caption=Environment class outline]
#Environment

import numpy as np
from numpy import genfromtxt

class Environment():

    def gen_env_from_csv(self, path):
        # read in .csv file (1, 0 ,0, 0, 1) as a numpy array.

    def horizontal_environment(self):
        # return array in the form [1, 0 ,0, 0, 1]
    
\end{lstlisting}

%%%%%%%%%%%%%%%%%%%%%%%%%%%%%%%%%%%%%%%%%%%%%%%%%%%%%%%%%%%%%%%%%%%%%%%%%%%%%%%%%%%%%%%%%%%%%%%%%%%

\subsection{Knowledge base (I'm not sure how to store things in a KB yet)} 

\begin{itemize}
    \item KR1: Should be able to store logic rules. (Will probably use strings to accomplish this).
 operations.
\end{itemize}


\begin{lstlisting}[language=Python, caption=Knowledge base class outline]
#Knowledge base

class KB():

    def add_rule(self):
        # add rule to KB

\end{lstlisting}



%%%%%%%%%%%%%%%%%%%%%%%%%%%%%%%%%%%%%%%%%%%%%%%%%%%%%%%%%%%%%%%%%%%%%%%%%%%%%%%%%%%%%%%%%%%%%%%%%%%

\subsection{Inference engine (Not completely sure how to implement this yet)}

\begin{itemize}
    \item IR1: Should be able to generate all possible worlds
    \item IR2: Check whether the facts in the KB agree with possible worlds by querying the KB.
    \item IR3: Should be able to execute the following logical operations:  $\neg$, $\wedge$ , $\vee$, $\Rightarrow$, $\Leftrightarrow$
    \item IR4: Use recursion operation from one time step to the next.
\end{itemize}



\begin{lstlisting}[language=Python, caption=Inference engine class outline]
#Inference engine

class Inference()

    def gen_worlds(self):
        #return possible worlds
        
    def infer(self):
        #check facts in KB against possible worlds

\end{lstlisting}





%%%%%%%%%%%%%%%%%%%%%%%%%%%%%%%%%%%%%%%%%%%%%%%%%%%%%%%%%%%%%%%%%%%%%%%%%%%%%%%%%%%%%%%%%%%%%%%%%%%

\end{document}
