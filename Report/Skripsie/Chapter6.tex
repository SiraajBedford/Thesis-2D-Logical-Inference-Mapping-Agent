%%%%%%%%%%%%%%%%%%%%%%%%%%%%%%%%%%%%%%%%%%%%%%%%%%%%%%%%%%%%%%%%%%%%%%%%%%%%%%%%%%%%%%%
% ELO 4,6
% Planning for Conclusion

% Chapter 6: Conclusion
% This is what we tried to do...
% This is what we gave in the whole report...
% List the objectives and say which metric you uwsed and say you got the following results...

% 6.1 Objective and Results Comparison

% Objective 1...

% Objective 2...


% 6.2 So what? (Mirros Chapter 1's problem statement)
% Includes work that you found out needs to be done.
% Future work 
% Recommendations
%%%%%%%%%%%%%%%%%%%%%%%%%%%%%%%%%%%%%%%%%%%%%%%%%%%%%%%%%%%%%%%%%%%%%%%%%%%%%%%%%%%%%%

\chapter{Conclusion} 
Shows it is possible. Just one step to make the framework viable.

\label{Conclusion}
% Expert systems
\section{Results and Objectives Comparison}
\section{Improvements and Project Extensions}

\begin{itemize}
	\item Multiple Agents for Faster Learning
	\item Map Learning Agent for 3-Dimensional Environments
	\item Karnaugh Map Solver for Minimum Logic Hardware
	\item Improving the Inference Engine with First-Order Logic
	\item Implementing predictions using Hopfield Networks
\end{itemize}



%Sense input via inference
%Get true propositions
%Build State Definitions from true propositions
%Build Environment from State Definitions 
%While Environment is being build, expand State Definitions by to adding vertex descriptions
%Represent Environment in KB with expanded state definitions
%Ask via inference which state the agent is in from expanded state definitions 






\newpage

About your question: It is absolutely possible to pass your skripsie if you have not successfully solved the whole problem.  My recommendation is:

\begin{itemize}
	\item Make sure that there is a certain core part of the skripsie that you have solved, implemented and evaluated thoroughly.  It seems that you have solved the inference part?  I'd then recommend that you show that you can identify the agent state with hand-designed state definitions.  If you can compare and evaluate the 3 different inference algorithms against each other, and demonstrate that inference works for different environments, it would be good.  
	\item For the part that you would not have completed (learning the states), it is important to show how far you got.  Make sure you describe your design of the solution, which aspects of it work and for which scenarios it works (if you can show that it works for a small environment, it would be great), as well as which aspects do not yet work.  If you can explain why the parts that do not work do not work, it would be good.
	\item Do not spend too much time on trying to solve the whole problem and then not hand in a well-written report or miss the hand-in deadline.  
	\item Review the ECSA outcomes, and make sure you address all of them in your project.  
\end{itemize}

You have put a lot of effort into your project and you do have a difficult skripsie; if you can show that you have successfully solved a part of this difficult project, you should have no problem to pass (disclaimer: it will be enough for me, but I cannot predict what the internal and external examiners will think).

I suggest you continue to work hard on your project, do not throw in the towel, and make sure you hand in a well-written report.  






















