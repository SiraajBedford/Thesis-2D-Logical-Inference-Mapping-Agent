%\setcounter{table}{0}
%\renewcommand{\thetable}{C.\arabic{table}}

\chapter{Appendix C: Code Extracts}
\label{chapter:code}

\section*{Knowledge Base Implementation}

\begin{lstlisting}[language=Python, caption=Knowledge Base Implementation.]
#Agent

class PropKB(KB):
    """A KB for propositional logic. Inefficient, with no indexing."""

    def __init__(self, sentence=None):
        super().__init__(sentence)
        self.clauses = []

    def tell(self, sentence):
        """Add the sentence's clauses to the KB."""
        self.clauses.extend(conjuncts(to_cnf(sentence)))

    def ask_generator(self, query):
        """Yield the empty substitution {} if KB entails query; else no results."""
        if tt_entails(Expr('&', *self.clauses), query):
            yield {}

    def ask_if_true(self, query):
        """Return True if the KB entails query, else return False."""
        for _ in self.ask_generator(query):
            return True
        return False

    def retract(self, sentence):
        """Remove the sentence's clauses from the KB."""
        for c in conjuncts(to_cnf(sentence)):
            if c in self.clauses:
                self.clauses.remove(c)
\end{lstlisting}
\label{listing:KB_implemnt}


\subsection*{Inference Algorithms Implementation}



\begin{lstlisting}[language=Python, caption=Model Checking Algorithm]
def tt_check_all(kb, alpha, symbols, model):
    """Auxiliary routine to implement tt_entails."""
    if not symbols:
        if pl_true(kb, model):
            result = pl_true(alpha, model)
            assert result in (True, False)
            return result
        else:
            return True
    else:
        P, rest = symbols[0], symbols[1:]
        return (tt_check_all(kb, alpha, rest, extend(model, P, True)) and
                tt_check_all(kb, alpha, rest, extend(model, P, False)))
\end{lstlisting}
\label{listing:m_checking}

\newpage

\begin{lstlisting}[language=Python, caption=Resolution Algorithm.]
def pl_resolution(kb, alpha):
    """
    [Figure 7.12]
    Propositional-logic resolution: say if alpha follows from KB.
    >>> pl_resolution(horn_clauses_KB, A)
    True
    """
    clauses = kb.clauses + conjuncts(to_cnf(~alpha))
    new = set()
    while True:
        n = len(clauses)
        pairs = [(clauses[i], clauses[j])
                 for i in range(n) for j in range(i + 1, n)]
        for (ci, cj) in pairs:
            resolvents = pl_resolve(ci, cj)
            if False in resolvents:
                return True
            new = new.union(set(resolvents))
        if new.issubset(set(clauses)):
            return False
        for c in new:
            if c not in clauses:
                clauses.append(c)
\end{lstlisting}
\label{listing:resolution}


\begin{lstlisting}[language=Python, caption=DPLL Algorithm,]
def dpll(clauses, symbols, model, branching_heuristic=no_branching_heuristic):
    """See if the clauses are true in a partial model."""
    unknown_clauses = []  # clauses with an unknown truth value
    for c in clauses:
        val = pl_true(c, model)
        if val is False:
            return False
        if val is None:
            unknown_clauses.append(c)
    if not unknown_clauses:
        return model
    P, value = find_pure_symbol(symbols, unknown_clauses)
    if P:
        return dpll(clauses, remove_all(P, symbols), extend(model, P, value), branching_heuristic)
    P, value = find_unit_clause(clauses, model)
    if P:
        return dpll(clauses, remove_all(P, symbols), extend(model, P, value), branching_heuristic)
    P, value = branching_heuristic(symbols, unknown_clauses)
    return (dpll(clauses, remove_all(P, symbols), extend(model, P, value), branching_heuristic) or
            dpll(clauses, remove_all(P, symbols), extend(model, P, not value), branching_heuristic))

\end{lstlisting}
\label{listing:DPLL}


\subsection*{Sensor Implementation}


\begin{lstlisting}[language=Python, caption=Agent Sensor Implmentation.]
def make_percept_sentence(self, percept_matrix, time):

        # Things perceived

        # Chain start
        #-------------------------------------------------
        if percept_matrix[0] == open_block_1:
            self.tell(percept_open_left_block(time))
            chain=percept_open_left_block(time)
            
        if percept_matrix[0] == not_open_block_1:
            self.tell(~percept_open_left_block(time))
            chain=~percept_open_left_block(time)
        #-------------------------------------------------

        if percept_matrix[1] == open_block_1:
            self.tell(percept_open_right_block(time))
            chain=chain & percept_open_right_block(time)

        if percept_matrix[2] == open_block_1:
            self.tell(percept_open_up_block(time))
            chain=chain & percept_open_up_block(time)
            
        if percept_matrix[3] == open_block_1:
            self.tell(percept_open_upper_left_block(time))
            chain=chain & percept_open_upper_left_block(time)
        
        if percept_matrix[4] == open_block_1:
            self.tell(percept_open_upper_right_block(time))
            chain=chain & percept_open_upper_right_block(time)

            
        if percept_matrix[5] == open_block_1:
            self.tell(percept_open_bottom_block(time))
            chain=chain & percept_open_bottom_block(time)
            # instead of having res, we can set a flag called 'flag=percept_open_bottom_block(time)' and then chain on other percept sentences

        if percept_matrix[6] == open_block_1:
            self.tell(percept_open_bottom_left_block(time))
            chain=chain & percept_open_bottom_left_block(time)
            
        if percept_matrix[7] == open_block_1:
            self.tell(percept_open_bottom_right_block(time))
            chain=chain & percept_open_bottom_right_block(time)
            
        # Things not perceived  

        if percept_matrix[1] == not_open_block_1:
            self.tell(~percept_open_right_block(time))
            chain=chain & ~percept_open_right_block(time)

        if percept_matrix[2] == not_open_block_1:
            self.tell(~percept_open_up_block(time))
            chain=chain & ~percept_open_up_block(time)
            
        if percept_matrix[3] == not_open_block_1:
            self.tell(~percept_open_upper_left_block(time))
            chain=chain & ~percept_open_upper_left_block(time)

        if percept_matrix[4] == not_open_block_1:
            self.tell(~percept_open_upper_right_block(time))
            chain=chain & ~percept_open_upper_right_block(time)
            
        if percept_matrix[5] == not_open_block_1:
            self.tell(~percept_open_bottom_block(time))
            chain=chain & ~percept_open_bottom_block(time)

        if percept_matrix[6] == not_open_block_1:
            self.tell(~percept_open_bottom_left_block(time))
            chain=chain & ~percept_open_bottom_left_block(time)

        if percept_matrix[7] == not_open_block_1:
            self.tell(~percept_open_bottom_right_block(time))
            chain=chain & ~percept_open_bottom_right_block(time)
            
        return chain 
\end{lstlisting}
\label{listing:sensor}


