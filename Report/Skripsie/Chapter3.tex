\chapter{System Design} 
\label{chapter:System_Design}

%%%%%%%%%%%%%%%%%%%%%%%%%%%%%%%%%%%%%%%%%%%%%
% ELO 3, 6
% Planning for System Design

% Chapter 3: System Design

% System Diagram with functional blocks 

% 3.1.1 Knowledge Base
% To store knowledge I could have used option 1,2 or 3, (List options at disposal) No indepth level. It isporetant you introduced the examiner tio the concepts in Chapter 2 to see the requirements for such a functional block. 

% 3.1.2 Functional Block 2 ...

% *Put intefaces between functional block, e.g. I2C, SPI for an electrical project. 

% 3.2 Metrics
% I want to measure whether I meet the objectives 1,2,3. This will say how/what I will measure. What can actualy be masued and what cannot be measured. Each metrics must prove that you can reach the requirement.
%%%%%%%%%%%%%%%%%%%%%%%%%%%%%%%%%%%%%%%%%%%%%


%%%%%%%%%%%%%%%%%%%%%%%%%%%%%%%%%%%%%%%%%%%%%


This chapter goes over in detail how and why the agent interacts with the environment. 


\begin{figure}[H]
    \centering
    \includegraphics[scale=.7]{Figures/main_architecture.PNG}
    \caption{General learning agent Figure 2.15 from \cite{russell2016artificial}} 
    \label{fig:agent}
\end{figure}

Fill the functional blocks with more "color"

Also put interfaces between blocks using arrows.

\section{Program Architecture}

Show agent in environment with robot in environment, actuators, remember agent controls the robot and the robot is in the physical environment.

Show all possible choices you could have made for reach of the functional blocks.

\subsection{Knowledge-based Agent}
		
\begin{figure}[H]
    \centering
    \includegraphics[scale=1]{Figures/Goal4.png}
    \caption{Logical inference in context}
    \label{fig:inference_logic}
\end{figure}
	
\subsection{Environment}



\subsection{Knowledge Base}



\subsection{Inferencer}

Inference in general: conclusion reached on the basis of evidence and reasoning.

Inference in machine learning: Inference refers to the process of using a trained machine learning algorithm to make a prediction.




\section{Metrics}

I want to measure whether I meet the of the objects.
This section says how and what I am going to measure.

List what you intend to measure and how do you intend to measure it.

You can match the metrics up to the objectives.

\section{Method}

Explain the experimental setup.
\subsection{Program Flow}