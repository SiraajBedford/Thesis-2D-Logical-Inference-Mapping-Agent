\chapter{System Design} 
\label{chapter:sys_design}

\begin{figure}[H]
    \centering
    \includegraphics[scale=.7]{Figures/main_architecture.png}
    \caption{Block diagram overview }
    \label{fig:sysdiag}
\end{figure}

This chapter goes over in detail how and why the agent interacts with the environment. 


\begin{figure}[H]
    \centering
    \includegraphics[scale=.7]{Figures/General_learning_agent.PNG}
    \caption{General learning agent Figure 2.15 from \cite{russell2016artificial}} 
    \label{fig:agent}
\end{figure}

Fill the functional blocks with more "color"

Also put interfaces between blocks using arrows.

\section{Program Architecture}

Show agent in environment with robot in environment, actuators, remember agent controls the robot and the robot is in the physical environment.

Show all possible choices you could have made for reach of the functional blocks.

\subsection{Agent}
		
	
\subsection{Environment}



\subsection{Knowledge Base}



\subsection{Inferencer}

Inference in general: conclusion reached on the basis of evidence and reasoning.

Inference in machine learning: Inference refers to the process of using a trained machine learning algorithm to make a prediction.

\section{Metrics}

I want to measure whether I meet the of the objects.
This section says how and what I am going to measure.

List what you intend to measure and how do you intend to measure it.

You can match the metrics up to the objectives.

\section{Method}

Explain the experimental setup.