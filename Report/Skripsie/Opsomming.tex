{\let\clearpage\relax \chapter{Uittreksel}}


%This project is aimed at developing a new solution to map a 2-dimensional environment using an agent that works using logic. The problem requires that the agent has no knowledge of its reference with respect to the environment and also has no knowledge of the meaning of the percepts it receives.
%
%The method we propose is to let the agent roam in the environment and discover the structure for itself. When roaming, the agent uses logical inference to form state transitions which can be used to represent the structure of the environment.
%
%The initial focus of the report is on necessary knowledge to understand the logic techniques for solving this problem. With this knowledge, the agent is designed, implemented and judged against the set objectives. The results indicate that the agent can map an environment having non-repeated unique states and can partially map 
%an environment having repeated unique states.



Hierdie projek probeer om 'n nuwe oplossing vir 'n  tweedimensionele omgewing in a kaart te bring met die behulp van 'n agent wat logika gebruik. Die agent het geen kennis van sy verwysing met betrekking van die omgewing nie en hy het ook geen kennis oor die persepsies se betekenis wat hy ontvang nie.

Die metode wat ons voorstel is om die agent in die omgewing te laat dwaal en self die struktuur te ontdek. Terwyl die agent in die omgewing dwaal, gebruik hy logiese afleiding om toestandoorgange te vorm dat kan gebruik word om die struktuur van die omgewing voor te stel.

Die aanvanklike fokus van die verslag is op die nodige kennis om die logiese tegnieke vir die oplossing van hierdie probleem te verstaan. Dan word die agent ontwerp, geïmplementeer en beoordeel volgens die gestelde doelstellings. Die resultate dui aan dat die agent 'n omgewing met nie-herhaalde unieke toestande kan karteer en gedeeltelik 'n omgewing met herhaalde unieke toestande karteer.